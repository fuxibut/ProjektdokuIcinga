% !TEX root = Projektdokumentation.tex

% Es werden nur die Abkürzungen aufgelistet, die mit \ac definiert und auch benutzt wurden. 
%
% \acro{VERSIS}{Versicherungsinformationssystem\acroextra{ (Bestandsführungssystem)}}
% Ergibt in der Liste: VERSIS Versicherungsinformationssystem (Bestandsführungssystem)
% Im Text aber: \ac{VERSIS} -> Versicherungsinformationssystem (VERSIS)

% Hinweis: allgemein bekannte Abkürzungen wie z.B. bzw. u.a. müssen nicht ins Abkürzungsverzeichnis aufgenommen werden
% Hinweis: allgemein bekannte IT-Begriffe wie Datenbank oder Programmiersprache müssen nicht erläutert werden,
%          aber ggfs. Fachbegriffe aus der Domäne des Prüflings (z.B. Versicherung)

% Die Option (in den eckigen Klammern) enthält das längste Label oder
% einen Platzhalter der die Breite der linken Spalte bestimmt.
\begin{acronym}[WWWW]
	\acro{PHP}{Hypertext Preprocessor; Skriptsprache zur Erstellung von Websites}
	\acro{CPU}{Central Processing Unit; Hauptprozessor}
	\acro{RAM}{Random-Access Memory; Arbeitsspeicher}
	\acro{VM}{Virtuelle Maschine}
	\acro{LAN}{Local Area Network}
	\acro{LDAP}{Lightweight Directory Access Protocol; Protokoll zur Abfrage von Verzeichnisdiensten}
	\acro{FQDN}{Full Qualified Domain Name; eindeutige Adresse eines Hosts im Netzwerk}
	\acro{HTTP}{Hypertext Transfer Protocol; Protokoll zur Übertragung von z.B. Websites}
	\acro{URL}{Uniform Resource Locator; hier: Internetadresse}
	\acro{ESXi}{Typ-1 Hypervisor von VMware}
\end{acronym}
