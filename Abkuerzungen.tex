% !TEX root = Projektdokumentation.tex

% Es werden nur die Abkürzungen aufgelistet, die mit \ac definiert und auch benutzt wurden. 
%
% \acro{VERSIS}{Versicherungsinformationssystem\acroextra{ (Bestandsführungssystem)}}
% Ergibt in der Liste: VERSIS Versicherungsinformationssystem (Bestandsführungssystem)
% Im Text aber: \ac{VERSIS} -> Versicherungsinformationssystem (VERSIS)

% Hinweis: allgemein bekannte Abkürzungen wie z.B. bzw. u.a. müssen nicht ins Abkürzungsverzeichnis aufgenommen werden
% Hinweis: allgemein bekannte IT-Begriffe wie Datenbank oder Programmiersprache müssen nicht erläutert werden,
%          aber ggfs. Fachbegriffe aus der Domäne des Prüflings (z.B. Versicherung)

% Die Option (in den eckigen Klammern) enthält das längste Label oder
% einen Platzhalter der die Breite der linken Spalte bestimmt.
% Please add the following required packages to your document preamble:
% \usepackage[table,xcdraw]{xcolor}
% If you use beamer only pass "xcolor=table" option, i.e. \documentclass[xcolor=table]{beamer}

\begin{table}[htb!]
\begin{tabular}{ll}
{\color[HTML]{000000} \textbf{UEFI}} & {\color[HTML]{000000} Unified Extensible Firmware Interface;   Nachfolger von BIOS}                           \\
{\color[HTML]{000000} \textbf{BIOS}} & {\color[HTML]{000000} Basic Input/Output System; Firmware eines Computers}                                    \\
{\color[HTML]{000000} \textbf{PHP}}  & {\color[HTML]{000000} Hypertext Preprocessor; Skriptsprache zur Erstellung von Websites}                      \\
{\color[HTML]{000000} \textbf{CPU}}  & {\color[HTML]{000000} Central Processing Unit; Hauptprozessor}                                                \\
{\color[HTML]{000000} \textbf{RAM}}  & {\color[HTML]{000000} Random-Access Memory; Arbeitsspeicher}                                                  \\
{\color[HTML]{000000} \textbf{VM}}   & {\color[HTML]{000000} Virtuelle Maschine}                                                                     \\
{\color[HTML]{000000} \textbf{LAN}}  & {\color[HTML]{000000} Local Area Network}                                                                     \\
{\color[HTML]{000000} \textbf{IDO}}  & {\color[HTML]{000000} Icinga Data Output}                                                                     \\
{\color[HTML]{000000} \textbf{LDAP}} & {\color[HTML]{000000} Lightweight Directory Access Protocol; Protokoll zur Abfrage von   Verzeichnisdiensten} \\
{\color[HTML]{000000} \textbf{FQDN}} & {\color[HTML]{000000} Full Qualified Domain Name; eindeutige Adresse eines Hosts im Netzwerk}                 \\
{\color[HTML]{000000} \textbf{HTTP}} & {\color[HTML]{000000} Hypertext Transfer Protocol; Protokoll zur Übertragung von z.B. Websites}               \\
{\color[HTML]{000000} \textbf{URL}}  & {\color[HTML]{000000} Uniform Resource Locator; hier: Internetadresse}                                        \\
{\color[HTML]{000000} \textbf{ESXi}} & {\color[HTML]{000000} Typ-1 Hypervisor von VMware}                                                            \\
{\color[HTML]{000000} \textbf{ISO}}  & {\color[HTML]{000000} Speicherabbild eines Dateisystems}                                                     
\end{tabular}
\end{table}