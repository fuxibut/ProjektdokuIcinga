% !TEX root = Projektdokumentation.tex
\section{Anhang}

\subsection{Berechnung des Stundensatzes eines Mitarbeitenden}
\label{app:Stundensatz}
Nach der Entgeldtabelle der IG Metall Bayern für Metall und Elektro\footnote{Vgl. \cite{Entgeldtabelle}} erhalten Angestellte der Entgeldgruppe EG 08 ein Monatsbruttogehalt von 3.800,00€. Zusätzlich dazu müssen ungefähr ein Fünftel für die Arbeitgeberanteile der Sozialversicherung hinzugerechnet werden.\footnote{Vgl. \cite{Personalkosten}} Aufsummiert ergibt das jährliche Personalkosten von 54.720,00€.

Eine Arbeitszeit von 35 Stunden pro Woche bedeuetet bei 52 Wochen eine Jahresarbeitszeit von 1820 Stunden. Abzüglich der Urlaubstage (30), Feiertage (ca. 11) und Fehltage durch Krankheit und Fortbildungen (ca. 15) ergibt das eine Jahresarbeitszeit von 1498 Stunden pro Jahr.

Werden die Jahrespersonalkosten von 54.720,00€ durch die Jahresarbeitszeit von 1498 Stunden dividiert, ergibt dies Stundenkosten für den Arbeitgeber in Höhe von 36,53€. Da es sich hierbei um eine Schätzung handelt, wird vereinfacht von 40,00€ pro Stunde ausgegangen.  

Für Auszubildende mit einer Ausbildungsvergütung in Höhe von 1.207,00€ \footnote{Vgl. \cite{EntgeldtabelleAzubis}} ergeben sich nach selber Kalkulation Personalkosten in Höhe von 10,57€, also vereinfacht 10,00€ pro Stunde.

\subsection{Detaillierte Zeitplanung}
\label{app:Zeitplanung}

\tabelleAnhang{ZeitplanungKomplett}

\input{Anhang/AnhangLastenheft.tex}
\clearpage

\subsection{Use Case-Diagramm}
\label{app:UseCase}
Use Case-Diagramme und weitere \acs{UML}-Diagramme kann man auch direkt mit \LaTeX{} zeichnen, siehe \zB \url{http://metauml.sourceforge.net/old/usecase-diagram.html}.
\begin{figure}[htb]
\centering
\includegraphicsKeepAspectRatio{UseCase.pdf}{0.7}
\caption{Use Case-Diagramm}
\end{figure}

\input{Anhang/AnhangPflichtenheft.tex}

\subsection{Datenbankmodell}
\label{app:Datenbankmodell}
ER-Modelle kann man auch direkt mit \LaTeX{} zeichnen, siehe \zB \url{http://www.texample.net/tikz/examples/entity-relationship-diagram/}.
\begin{figure}[htb]
\centering
\includegraphicsKeepAspectRatio{database.pdf}{1}
\caption{Datenbankmodell}
\end{figure}
\clearpage

\input{Anhang/AnhangEntwuerfe.tex}
\clearpage
\subsection{Screenshots}
\label{Screenshots}

\begin{figure}[htb]
\centering
\includegraphicsKeepAspectRatio{screen_vmnetwork.jpg}{0.9}
\caption{Netzwerkkonfiguration in VMware ESXi. Rechts der physische Netzwerkport, links die virtuellen Ports der virtuellen Maschinen. Beides verbunden durch einen virtuellen Switch (mitte).}
\label{screen:vmnetwork}
\end{figure}

\begin{figure}[htb]
\centering
\includegraphicsKeepAspectRatio{screen_vmcreation.jpg}{0.9}
\caption{Anpassung der Parameter bei Erstellung einer virtuellen Maschine in VMware ESXi}
\label{screen:vmcreation}
\end{figure}
\clearpage

\begin{figure}[!htb]
\centering
\includegraphicsKeepAspectRatio{screen_mysqlsecure.jpg}{0.9}
\caption{Ausführen des Skripts \glqq{}mysql\_{}secure\_{}installation\grqq{} zur Absicherung eines MySQL-Systems}
\label{screen:mysqlsecure}
\end{figure}

\begin{figure}[!htb]
\centering
\includegraphicsKeepAspectRatio{screen_phperror.jpg}{1}
\caption{Angezeigter PHP-Fehler nach Installation des Icinga-Webfrontends}
\label{screen:phperror}
\end{figure}

\begin{figure}[!htb]
\centering
\includegraphicsKeepAspectRatio{screen_konfigassistent.jpg}{1}
\caption{Willkommens-Bildschirm des Konfigurationsassistenten}
\label{screen:konfigassistent}
\end{figure}

\begin{figure}[!htb]
\centering
\includegraphicsKeepAspectRatio{screen_userdb.jpg}{1}
\caption{Einrichtung der Datenbank für Webfrontend-Benutzer}
\label{screen:userdb}
\end{figure}

\begin{figure}[!htb]
\centering
\includegraphicsKeepAspectRatio{screen_landingpage.jpg}{1}
\caption{Startseite von \glqq{}Icinga 2\grqq{} nach der Erstkonfiguration. Der Server, auf dem die Instanz von \glqq{}Icinga 2\grqq{} läuft, ist automatisch als erster Server hinzugefügt}
\label{screen:landingpage}
\end{figure}

\begin{figure}[!htb]
\centering
\includegraphicsKeepAspectRatio{screen_newservers.jpg}{1}
\caption{Zwei neue Server werden durch Anhängen dieser Zeilen an \code{/etc/icinga2/conf.d/hosts.conf} dem Monitoring hinzugefügt}
\label{screen:newservers}
\end{figure}

\begin{figure}[!htb]
\centering
\includegraphicsKeepAspectRatio{screen_loaddefinition.jpg}{0.5}
\caption{Beispiel für ein definiertes \glqq{}CheckCommand\grqq{}-Objekt}
\label{screen:loaddefinition}
\end{figure}

\begin{figure}[!htb]
\centering
\includegraphicsKeepAspectRatio{screen_service.png}{0.5}
\caption{Beispiel für einen definierten Dienst}
\label{screen:service}
\end{figure}

\begin{figure}[!htb]
\centering
\includegraphicsKeepAspectRatio{screen_webserver.png}{0.5}
\caption{Detailansicht eines überprüften Dienstes; hier die Webserverüberwachung}
\label{screen:webserver}
\end{figure}

\begin{figure}[!htb]
\centering
\includegraphicsKeepAspectRatio{screen_dashboard.png}{0.9}
\caption{Selbsterstelltes Dashboard}
\label{screen:dashboard}
\end{figure}
\input{Anhang/AnhangDoc.tex}
\clearpage
\input{Anhang/AnhangTest.tex}

\subsection{Klasse: ComparedNaturalModuleInformation}
\label{app:CNMI}
Kommentare und simple Getter/Setter werden nicht angezeigt.
\lstinputlisting[language=php, caption={Klasse: ComparedNaturalModuleInformation}]{Listings/cnmi.php}
\clearpage

\subsection{Klassendiagramm}
\label{app:Klassendiagramm}
Klassendiagramme und weitere \acs{UML}-Diagramme kann man auch direkt mit \LaTeX{} zeichnen, siehe \zB \url{http://metauml.sourceforge.net/old/class-diagram.html}.
\begin{figure}[htb]
\centering
\includegraphicsKeepAspectRatio{Klassendiagramm.pdf}{1}
\caption{Klassendiagramm}
\end{figure}
\clearpage

\input{Anhang/AnhangBenutzerDoku.tex}
