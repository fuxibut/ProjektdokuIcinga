\subsection{Screenshots}
\label{Screenshots}

\begin{figure}[htb]
\centering
\includegraphicsKeepAspectRatio{screen_vmnetwork.jpg}{0.9}
\caption{Netzwerkkonfiguration in VMware ESXi. (Rechts physische Netzwerkport, links virtuelle Ports der VMs, mitte ein virtueller Switch)}
\label{screen:vmnetwork}
\end{figure}

\begin{figure}[htb]
\centering
\includegraphicsKeepAspectRatio{screen_vmcreation.jpg}{0.9}
\caption{Parameteranpassung bei Erstellung einer virtuellen Maschine in VMware ESXi}
\label{screen:vmcreation}
\end{figure}
\clearpage

\begin{figure}[!htb]
\centering
\includegraphicsKeepAspectRatio{screen_mysqlsecure.jpg}{0.9}
\caption{Das Skript \glqq{}mysql\_{}secure\_{}installation\grqq{} zur Absicherung eines MySQL-Systems}
\label{screen:mysqlsecure}
\end{figure}

\begin{figure}[!htb]
\centering
\includegraphicsKeepAspectRatio{screen_phperror.jpg}{1}
\caption{PHP-Fehler nach Installation des Icinga-Webfrontends}
\label{screen:phperror}
\end{figure}

\begin{figure}[!htb]
\centering
\includegraphicsKeepAspectRatio{screen_konfigassistent.jpg}{1}
\caption{Willkommens-Bildschirm des Konfigurationsassistenten}
\label{screen:konfigassistent}
\end{figure}

\begin{figure}[!htb]
\centering
\includegraphicsKeepAspectRatio{screen_userdb.jpg}{1}
\caption{Einrichtung der Datenbank für Webfrontend-Benutzer}
\label{screen:userdb}
\end{figure}

\begin{figure}[!htb]
\centering
\includegraphicsKeepAspectRatio{screen_landingpage.jpg}{1}
\caption{Startseite von \glqq{}Icinga 2\grqq{} nach der Erstkonfiguration. Der Server, auf dem die Instanz von \glqq{}Icinga 2\grqq{} läuft, ist automatisch als erster Server hinzugefügt}
\label{screen:landingpage}
\end{figure}

\begin{figure}[!htb]
\centering
\includegraphicsKeepAspectRatio{screen_newservers.png}{1}
\caption{Konfiguration in \code{/etc/icinga2/conf.d/hosts.conf} um zwei neue Server dem Monitoring hinzuzufügen}
\label{screen:newservers}
\end{figure}

\begin{figure}[!htb]
\centering
\includegraphicsKeepAspectRatio{screen_loaddefinition.png}{0.5}
\caption{Beispiel für ein definiertes \glqq{}CheckCommand\grqq{}-Objekt}
\label{screen:loaddefinition}
\end{figure}

\begin{figure}[!htb]
\centering
\includegraphicsKeepAspectRatio{screen_service.png}{0.5}
\caption{Beispiel für einen definierten Dienst}
\label{screen:service}
\end{figure}

\begin{figure}[!htb]
\centering
\includegraphicsKeepAspectRatio{screen_webserver.png}{0.5}
\caption{Detailansicht eines überprüften Dienstes; hier die Webserverüberwachung}
\label{screen:webserver}
\end{figure}

\begin{figure}[!htb]
\centering
\includegraphicsKeepAspectRatio{screen_dashboard.png}{0.9}
\caption{Selbsterstelltes Dashboard}
\label{screen:dashboard}
\end{figure}