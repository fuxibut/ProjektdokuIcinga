\subsection{Screenshots}
\label{Screenshots}

\begin{figure}[htb]
\centering
\includegraphicsKeepAspectRatio{screen_vmnetwork.jpg}{0.9}
\caption{Netzwerkkonfiguration in VMware ESXi. Rechts der physische Netzwerkport, links die virtuellen Ports der virtuellen Maschinen. Beides verbunden durch einen virtuellen Switch (mitte).}
\label{screen:vmnetwork}
\end{figure}

\begin{figure}[htb]
\centering
\includegraphicsKeepAspectRatio{screen_vmcreation.jpg}{0.9}
\caption{Anpassung der Parameter bei Erstellung einer virtuellen Maschine in VMware ESXi}
\label{screen:vmcreation}
\end{figure}
\clearpage

\begin{figure}[!htb]
\centering
\includegraphicsKeepAspectRatio{screen_mysqlsecure.jpg}{0.9}
\caption{Ausführen des Skripts \glqq{}mysql\_{}secure\_{}installation\grqq{} zur Absicherung eines MySQL-Systems}
\label{screen:mysqlsecure}
\end{figure}

\begin{figure}[!htb]
\centering
\includegraphicsKeepAspectRatio{screen_phperror.jpg}{1}
\caption{Angezeigter PHP-Fehler nach Installation des Icinga-Webfrontends}
\label{screen:phperror}
\end{figure}

\begin{figure}[!htb]
\centering
\includegraphicsKeepAspectRatio{screen_konfigassistent.jpg}{1}
\caption{Willkommens-Bildschirm des Konfigurationsassistenten}
\label{screen:konfigassistent}
\end{figure}

\begin{figure}[!htb]
\centering
\includegraphicsKeepAspectRatio{screen_userdb.jpg}{1}
\caption{Einrichtung der Datenbank für Webfrontend-Benutzer}
\label{screen:userdb}
\end{figure}

\begin{figure}[!htb]
\centering
\includegraphicsKeepAspectRatio{screen_landingpage.jpg}{1}
\caption{Startseite von \glqq{}Icinga 2\grqq{} nach der Erstkonfiguration. Der Server, auf dem die Instanz von \glqq{}Icinga 2\grqq{} läuft, ist automatisch als erster Server hinzugefügt}
\label{screen:landingpage}
\end{figure}

\begin{figure}[!htb]
\centering
\includegraphicsKeepAspectRatio{screen_newservers.jpg}{1}
\caption{Zwei neue Server werden durch Anhängen dieser Zeilen an \code{/etc/icinga2/conf.d/hosts.conf} dem Monitoring hinzugefügt}
\label{screen:newservers}
\end{figure}