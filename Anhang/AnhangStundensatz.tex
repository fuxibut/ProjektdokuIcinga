
\subsection{Berechnung des Stundensatzes von Mitarbeitenden}
\label{app:Stundensatz}
Nach der Entgeldtabelle der IG Metall Bayern für Metall und Elektro\footnote{Vgl. \cite{Entgeldtabelle}} erhalten Angestellte der Entgeldgruppe EG 08 ein Monatsbruttogehalt von \eur{3.800,00}. Zusätzlich dazu müssen ungefähr ein Fünftel für die Arbeitgeberanteile der Sozialversicherung und Mehrleistungen wie Urlaubsgeld hinzugerechnet werden.\footnote{Vgl. \cite{Personalkosten}} Aufsummiert ergibt das jährliche Personalkosten von \eur{54.720,00}.
\begin{eqnarray}
\eur{3.800,00} \cdot 1,2 \cdot 12 \mbox{ m} = \eur{54.720,00} \mbox{ /a}
\end{eqnarray}
Eine Arbeitszeit von 35 Stunden pro Woche bedeuetet bei 52 Wochen eine Jahresarbeitszeit von 1820 Stunden. Abzüglich der Urlaubstage (30), Feiertage (ca. 11) und Fehltage durch Krankheit und Fortbildungen (ca. 15) ergibt das eine Jahresarbeitszeit von 1498 Stunden pro Jahr.
\begin{eqnarray}
35 \mbox{h/w} \cdot 52 \mbox{ m} = 1820 \mbox{ h/a} \\
1820 \mbox{ h/a} - [(30+11+15) \mbox{d} \cdot 35 \mbox{h/d}] = 1498 \mbox{ h/a}
\end{eqnarray}
Werden die Jahrespersonalkosten von \eur{54.720,00} durch die Jahresarbeitszeit von 1498 Stunden dividiert, ergibt dies Stundenkosten für den Arbeitgeber in Höhe von \eur{36,53}. Da es sich hierbei um eine Schätzung handelt, wird vereinfacht von \eur{40,00} pro Stunde ausgegangen.  
\begin{eqnarray}
\eur{54.720,00} /a \div 220 \mbox{ h/a} = \eur{36,53} /h \approx \eur{40,00} /h 
\end{eqnarray}
Für Auszubildende mit einer Ausbildungsvergütung in Höhe von \eur{1.207,00} \footnote{Vgl. \cite{EntgeldtabelleAzubis}} ergeben sich nach selber Kalkulation Personalkosten in Höhe von \eur{10,57}, vereinfacht \eur{10,00} pro Stunde.