\subsection{Installationsskript für Ubuntu 18.04}
\label{app:skript}

\# Systemupdate \\
apt update\\
apt dist-upgrade\\

\# Grundinstallation icinga und MySQL\\
apt install icinga2 icingacli monitoring-plugins mysql-server mysql-client\\
mysql\_{}secure\_{}installation\\

\# Installation Icinga IDO und Web-Interface\\
apt install icinga2-ido-mysql\\
apt install icingaweb2\\

\# Klonen Github-Repository\\
cd /usr/share\\
mv icingaweb2 old.icingaweb2\\
git config --global http.proxy http://webproxy1.kuka.int.kuka.com:80\\
git clone https://github.com/Icinga/icingaweb2.git\\

\# Installation fehlendes PHP-Plugin\\
apt install php7.2-curl\\
systemctl restart apache2\\

\# Hinzufügen MySQL User für icingaweb2\\
mysql -e "CREATE USER 'icingaweb'@'localhost' IDENTIFIED BY '•••';"\\
mysql -e "GRANT ALL PRIVILEGES ON *.* TO 'icingaweb'@'localhost';"\\
mysql -e "FLUSH PRIVILEGES;"\\

\# Einrichtungstoken erstellen\\
icinga2 feature enable command ido-mysql\\
icingacli setup token create\\
systemctl restart icinga2\\