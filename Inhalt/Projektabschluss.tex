% !TEX root = ../Projektdokumentation.tex
\section{Projektabschluss} 
\label{sec:projektabschluss}

\subsection{Evaluationsergebnisse}
\label{sec:Evaluationsergebnisse}
\glqq{}Icinga 2\grqq{} konnte alle definierten Anforderungen erfüllen. Die einfache Erweiterbarkeit durch Plugins, sowie die Verfügbarkeit vieler verschiedener Erweiterungen, bedingen die enorme Flexibilität dieser Monitoringlösung. Das wichtigste Kriterium, die Überwachung der Systemparameter, ist dadurch in hohem Maße erfüllt. Die Administration wirkt zwar anfangs kompliziert, durch Anpassung der Konfigurationsdateien können Routineaufgaben wie das Hinzufügen neuer Server aber leicht automatisiert werden. Da bereits Arbeitskräfte im Unternehmen mit der Vorgängerversion gearbeitet haben, fällt die Komplexität des Systems weniger ins Gewicht.

Neben der hohen funktionalen Flexibilität sprechen auch die weiteren Vorteile einer Open-Source Lösung für \glqq{}Icinga 2\grqq{}, wie entfallende Lizenzkosten oder die erhöhte Sicherheit. Freie Software kann außerdem länger in Unternehmen eingesetzt werden, da die Abhängigkeit zu einem bestimmten Hersteller für Sicherheitsupdates oder Ähnlichem entfällt. Es wird an dieser Stelle also eine klare Empfehlung für den Einsatz von \glqq{}Icinga 2\grqq{} innerhalb der KUKA ausgesprochen; das Projektziel (siehe \ref{sec:Projektziel} \nameref{sec:Projektziel}) wurde erreicht.

\subsection{Abnahme}
\label{sec:Abnahme}
Das Projekt wurde im Rahmen einer Videokonferenz an den Projektbeauftragten präsentiert und übergeben.

\subsection{Lessons Learned}
\label{sec:LessonsLearned}
Das in \ref{sec:InstallationIcinga} \nameref{sec:InstallationIcinga} beschriebene Fehlerbild, welches nur durch Klonen des aktuellen GitHub-Archivs beseitigt werden konnte, kostete viel Zeit. Zukünftig sollten die Versionsnummern von heruntergeladener Software auf ihre Aktualität überprüft werden, um derartigen Fehlern vorzubeugen.

\subsection{Ausblick}
\label{sec:Ausblick}
Es folgt zunächst ein abteilungsinterner Rollout von \glqq{}Icinga 2\grqq{}. Sollte sich die Lösung auch in der Praxis als tauglich herausstellen, werden nach und nach die restlichen Abteilungen an das System angebunden.