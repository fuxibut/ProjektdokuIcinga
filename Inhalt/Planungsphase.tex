% !TEX root = ../Projektdokumentation.tex
\section{Planungsphase} 
\label{sec:Planungsphase}

\subsection{Ist-Analyse} 
\label{sec:IstAnalyse}
Zu Projektbeginn existiert im Unternehmen keine einheitliche Lösung zur Serverüberwachung. Für den Großteil der Server am Standort Augsburg wird der \glqq Advanced Host Monitor\grqq{} eingesetzt. Diese Lösung erhält trotz ihres Alters noch regelmäßige Updates, aufgrund eines veralteten User Interfaces, einem beschränktem Funktionsumfang sowie der nicht mehr ausreichenden Leistungsfähigkeit, wird seit mehreren Jahren der Wunsch nach einem neuen Monitoring-System geäußert.

Die Entwicklungsabteilungen überwachen die von ihnen betreuten Server mit einer auf \glqq Elasticsearch\grqq{} und \glqq Kibana\grqq{} basierenden Lösung. Um Know-how zu bündeln und Personalkosten zu sparen, signalisierten die Verantwortlichen eine Bereitschaft zur Zusammenlegung des Servermonitoring.

Am Standort Bremen wird seit mehreren Jahren auf die Open-Source-Anwendung \glqq Icinga\grqq{} gesetzt. Die Systembetreuenden haben mit dieser Lösung positive Erfahrungen gemacht, und loben insbesondere die einfache Erweiterbarkeit durch Plugins, sowie die Möglichkeit ohne Aufwand optisch ansprechende Dashboards zu erstellen. Aufgrund der Komplexität dieser Software und einem anstehenden Update auf \glqq Icinga 2\grqq{} wurde auch von dieser Seite der Wunsch nach einem einheitlichen und zentral verwaltetem Servermonitoring geäußert.

\subsection{Soll-Analyse} 
\label{sec:SollAnalyse}

\subsubsection{Befragung der Fachabteilungen}
\label{sec:BefragungFachabteilungen}
Es wurden die entsprechenden Verantwortlichen via E-Mail, Telefonaten und Meetings zu den betreuten Servern und deren Monitoring befragt. Hierbei zeigten sich große Überschneidungen bei den Serverumgebungen sowie den Anforderungen an deren Überwachung, was eine Zusammenlegung logisch erscheinen lässt.

Über alle Abteilungen hinweg werden verschiedenste Windows-Versionen sowie Linux-Distributionen eingesetzt, weswegen ausschließlich ein plattformunabhängiges Monitoring-System wie \glqq Icinga 2\grqq{} eingesetzt werden kann. Die zu überwachenden Parameter beziehungsweise Dienste ähneln sich zwischen den Abteilungen ebenfalls sehr stark: Neben Leistungsmetriken wie Prozessorauslastung, Arbeitsspeicherbelegung oder freiem Massenspeicher wird viel Wert auf die Überprüfung der Erreichbarkeit im Netzwerk gelegt. Aufgrund der zunehmenden Verbreitung von webbasierten Anwendungen soll auch der Betrieb von Webservern überwacht werden. Weiterhin wurde der Wunsch nach einer ansprechenden grafischen Aufbereitung geäußert.

\subsubsection{Kriterienkatalog}
\label{sec:Kriterienkatalog}
Auf Basis der Befragungen wurde folgender Kriterienkatalog aufgestellt:
\begin{itemize}
	\item Überwachbare Betriebssysteme: Alle gängigen Windows Server Versionen und Linux-Distributionen
	\item Überwachbare Systemparameter: CPU-Auslastung, RAM-Belegung, Massenspeicherbelegung
	\item Überwachung von Webservern
	\item Ansprechende und anpassbare grafische Aufbereitung
\end{itemize}

\subsection{Wirtschaftlichkeitsanalyse}
\label{sec:Wirtschaftlichkeitsanalyse}

\subsubsection{Projektkosten}
\label{sec:Projektkosten}
Die Kosten, die durch das Projekt verursacht werden, setzen sich sowohl aus Personal- als auch aus Ressourcenkosten zusammen. Die Berechnung der (fiktiven) Stundensätze findet sich im \Anhang{app:Stundensatz}.

Die Kosten für die 35-Stunden-Woche des Auszubildenden belaufen sich auf \eur{350}. Die Arbeitszeit, die bei mitarbeitenden Personen zur Durchführung des Projekts angefallen ist (z.B. für Befragungen, Konfiguration der Netzwerkkomponenten, Anpassungen der Firewall für Updates, Abnahme) wurde auf vier Stunden geschätzt. Dafür fielen Personalkosten in Höhe von \eur{160} an. Die für das Projekt eingesetzte Hardware ist bereits abgeschrieben, und wird mit einer Pauschale von \eur{360} pro Jahr verrechnet. Auf die Projektdauer von fünf Tagen ergibt das Betriebskosten von \eur{4,93}.

Eine Aufstellung der Projektosten befindet sich in Tabelle~\ref{tab:Kostenaufstellung}. Sie betragen insgesamt \eur{514,93}.
\tabelle{Kostenaufstellung Projekt}{tab:Kostenaufstellung}{Kostenaufstellung.tex}

\subsubsection{Betriebskosten des Monitorings}
\label{sec:BetriebskostenMonitoring}
Die Betriebskosten für ein Monitoringsystem umfassen Personalkosten für Wartungsarbeiten und Anpassungen, sowie die Kosten die für zwei redundant ausgelegte Hardware-Server anfallen. Diese werden benötigt, da das Monitoring nicht innerhalb der Virtualisierungsumgebung betrieben werden sollte, um eine funktionierende Serverüberwachung auch bei Ausfall der VM-Infrastruktur zu gewährleisten.

Für das Einpflegen neuer Server oder Funktionen wird eine Arbeitsstunde pro Woche veranschlagt. Weiterhin wird der Arbeitsaufwand für Updates und Fehlerbehebungen nach Erfahrungsberichten auf acht Stunden pro Quartal geschätzt. Auf ein Jahr summiert ergibt beides einen Arbeitsaufwand von insgesamt 84 Stunden.

Die geschätzten Kosten für einen Hardware-Server belaufen sich auf \eur{4.000} jährlich. Die Gesamtkosten für den Betrieb einer Monitoring-Instanz belaufen sich, wie in Tabelle~\ref{tab:Monitoringkosten} dargelegt, auf \eur{11.360,00} pro Jahr.
\tabelle{Kostenaufstellung Monitoring}{tab:Monitoringkosten}{Monitoringkosten.tex}

\subsubsection{Amortisationsdauer}
\label{sec:Amortisationsdauer}
Sollte die Evaluation durch diese Projektarbeit ergeben, dass \glqq{}Icinga 2\grqq{} für den firmenweiten Einsatz geeignet ist, werden drei momentan im Betrieb befindliche Monitoringinstanzen (siehe \ref{sec:IstAnalyse} \nameref{sec:IstAnalyse}) durch ein zentrales \glqq{}Icinga 2\grqq{}-System ersetzt (siehe \ref{sec:Projektziel} \nameref{sec:Projektziel}). Die Betriebskosten für die derzeitigen Überwachungssysteme entsprechen etwa den in \ref{sec:BetriebskostenMonitoring} \nameref{sec:BetriebskostenMonitoring} berechneten \eur{11.360,00} pro Jahr; die jährliche Einsparung, wenn statt drei Systemen nur noch eines betrieben wird, beläuft sich auf:
\begin{eqnarray}
\eur{11.360,00} \cdot 2 = \eur{22.720,00}
\end{eqnarray}
Für die Ersteinrichtung des neuen Systems werden etwa zwei Wochen benötigt, also insgesamt 70 Arbeitsstunden. Das verursacht einmalige Kosten in Höhe von:
\begin{eqnarray}
70 \mbox{h} \cdot \eur{40,00} \mbox{/h} = \eur{2.800}
\end{eqnarray}
Die Amortisationszeit beträgt somit:
\begin{eqnarray}
\frac{\eur{2.800}}{\eur{22.720,00}\mbox{/a}} \approx 0,123 \mbox{ Jahre} \approx 45 \mbox{ Tage}
\end{eqnarray}
\setcounter{equation}{0}