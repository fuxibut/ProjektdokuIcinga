% !TEX root = ../Projektdokumentation.tex
\section{Planungsphase} 
\label{sec:Planungsphase}

\subsection{Ist-Analyse} 
\label{sec:IstAnalyse}
Zu Projektbeginn existiert im Unternehmen keine einheitliche Lösung zur Serverüberwachung. Für den Großteil der Server am Standort Augsburg wird der \glqq Advanced Host Monitor\grqq{} eingesetzt. Diese Lösung erhält trotz ihres Alters noch regelmäßige Updates, aufgrund des veralteten User Interfaces und einem beschränktem Funktionsumfang, sowie der nicht ausreichenden Leistungsfähigkeit, wird seit mehreren Jahren der Wunsch nach einem neuen Monitoring-System geäußert.

Die Entwicklungsabteilungen überwachen die von ihnen betreuten Server mit einer auf \glqq Elasticsearch\grqq{} und \glqq Kibana\grqq{} basierenden Lösung. Um Know-how zu bündeln und Personalkosten zu sparen, signalisierten die Verantwortlichen eine Bereitschaft zur Zusammenlegung des Servermonitoring.

Am Standort Bremen wird seit mehreren Jahren auf die Open-Source-Anwendung \glqq Icinga\grqq{} gesetzt. Die Systembetreuer haben mit dieser Lösung positive Erfahrungen gemacht, und loben insbesondere die einfache Erweiterbarkeit durch Plugins, sowie die Möglichkeit ohne Aufwand optisch ansprechende Dashboards zu erstellen. Aufgrund der Komplexität dieser Software und einem anstehenden Update auf \glqq Icinga 2\grqq{} wurde auch von dieser Seite der Wunsch nach einem einheitlichen und zentral verwaltetem Servermonitoring geäußert.

\subsection{Soll-Analyse} 
\label{sec:SollAnalyse}

\subsubsection{Befragung der Fachabteilungen}
\label{sec:BefragungFachabteilungen}
Es wurden die entsprechenden Verantwortlichen via E-Mail, Telefonaten und Meetings zu den betreuten Servern und deren Monitoring befragt. Hierbei zeigten sich große Überschneidungen bei den Serverumgebungen sowie den Anforderungen an deren Überwachung, was eine Zusammenlegung logisch erscheinen lässt.

Über alle Abteilungen hinweg werden verschiedenste Windows-Versionen sowie Linux-Distributionen eingesetzt, weswegen ausschließlich ein plattformunabhängiges Monitoring-System wie \glqq Icinga\grqq{} eingesetzt werden kann. Zwar sind die meisten Server virtualisiert, allerdings werden gerade bei Datenbanken auch dedizierte Rechner eingesetzt, welche selbstverständlich ebenfalls überwacht werden müssen.

Die zu überwachenden Parameter beziehungsweise Dienste ähneln sich zwischen den Abteilungen ebenfalls sehr stark. Neben Leistungsmetriken wie Prozessorauslastung, Arbeitsspeicherbelegung oder freiem Massenspeicher wird viel Wert auf die Überprüfung der Erreichbarkeit im Netzwerk gelegt. Aufgrund der zunehmenden Verbreitung von webbasierten Anwendungen soll auch der Betrieb von Webservern überwacht werden. Weiterhin wurde der Wunsch nach einer grafischen Aufbereitung geäußert.

\subsubsection{Kriterienkatalog}
\label{sec:Kriterienkatalog}
Auf Basis der Befragungen wurde folgender Kriterienkatalog aufgestellt:
\begin{itemize}
	\item Überwachbare Betriebssysteme
	\begin{itemize}
		\item Alle aktuellen Windows Server Versionen
		\item Alle gängigen Linux-Distributionen
	\end{itemize}
	\item Überwachbare Systemparameter
	\begin{itemize}
		\item CPU-Auslastung
		\item RAM-Belegung
		\item Massenspeicherbelegung
	\end{itemize}
	\item Überwachung von Webservern
	\item Ansprechende und anpassbare grafische Aufbereitung
\end{itemize}

\subsection{Wirtschaftlichkeitsanalyse}
\label{sec:Wirtschaftlichkeitsanalyse}

\subsubsection{Projektkosten}
\label{sec:Projektkosten}
Die Kosten, die durch das Projekt verursacht werden, setzen sich sowohl aus Personal- als auch aus Ressourcenkosten zusammen. Die Berechnung der Stundensätze findet sich in \ref{app:Stundensatz} \nameref{app:Stundensatz}.

Die Kosten für die 35-Stunden-Woche des Auszubildenden belaufen sich auf \eur{350}. Die Arbeitszeit, die bei mitarbeitenden Personen zur Durchführung des Projekts angefallen ist (z.B. für Befragungen, Konfiguration der Netzwerkkomponenten, Anpassungen der Firewall für Updates, Abnahme) wurde auf vier Stunden geschätzt. Dafür fielen Personalkosten in Höhe von \eur{160} an. Die für das Projekt eingesetzte Hardware ist bereits abgeschrieben, und wird mit einer Pauschale von \eur{360} pro Jahr verrechnet. Auf die Projektdauer von fünf Tagen ergibt das Betriebskosten von \eur{4,93}.

Eine Aufstellung der Kosten befindet sich in Tabelle~\ref{tab:Kostenaufstellung}. Sie betragen insgesamt \eur{514,93}.
\tabelle{Kostenaufstellung}{tab:Kostenaufstellung}{Kostenaufstellung.tex}

\subsubsection{Betriebskosten des Monitorings}
\label{sec:BetriebskostenMonitoring}
\begin{itemize}
	\item Betriebskosten Server
	\item Arbeitskosten 1 Mitarbeiter
\end{itemize}

\subsubsection{Amortisationsdauer}
\label{sec:Amortisationsdauer}
\begin{itemize}
	\item Einsparung 2 mal Betriebskosten
	\item Personalkosten für Einrichtung
\end{itemize}

\paragraph{Beispielrechnung (verkürzt)}
Bei einer Zeiteinsparung von 10 Minuten am Tag für jeden der 25 Anwender und 220 Arbeitstagen im Jahr ergibt sich eine gesamte Zeiteinsparung von 
\begin{eqnarray}
25 \cdot 220 \mbox{ Tage/Jahr} \cdot 10 \mbox{ min/Tag} = 55000 \mbox{ min/Jahr} \approx 917 \mbox{ h/Jahr} 
\end{eqnarray}

Dadurch ergibt sich eine jährliche Einsparung von 
\begin{eqnarray}
917 \mbox{h} \cdot \eur{(25 + 15)}{\mbox{/h}} = \eur{36680}
\end{eqnarray}
\setcounter{equation}{0}
Die Amortisationszeit beträgt also $\frac{\eur{2739,20}}{\eur{36680}\mbox{/Jahr}} \approx 0,07 \mbox{ Jahre} \approx 4 \mbox{ Wochen}$.