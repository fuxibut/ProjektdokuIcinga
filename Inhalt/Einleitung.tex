% !TEX root = ../Projektdokumentation.tex
\section{Einleitung}
\label{sec:Einleitung}

\subsection{Projektumfeld} 
\label{sec:Projektumfeld}
Die KUKA AG ist ein international tätiges Unternehmen in der Maschinenbau- und Automatisierungsbranche mit rund 14.200 Mitarbeitenden. Zum Produktportfolio zählen neben Industrierobotern auch die Planung und der Bau von Produktionsstraßen.

Die IT-Infrastruktur am Standort Augsburg besteht aus ca. 1.200 größtenteils virtualisierten Servern; an anderen Standorten befindet sich vereinzelt eine kleinere Anzahl an Servern. Es kommen alle gängigen Betriebssysteme zum Einsatz. Dieses Projekt wurde durch die Abteilung \glqq Datacenter \& Network\grqq{}, die den Betrieb der Rechenzentren sowie des internen IT-Netzwerks der KUKA AG überwacht und koordiniert, in Auftrag gegeben.

\subsection{Projektziel} 
\label{sec:Projektziel}
Momentan wird kein einheitliches Monitoring der IT-Systeme durchgeführt. Die verschiedenen Standorte und Abteilungen setzen auf unterschiedliche und teils veraltete Lösungen; zur besseren Administration sollen diese durch ein einheitliches und zentrales System ersetzt werden.

Es soll evaluiert werden, ob die freie Monitoringsoftware \glqq Icinga 2\grqq{} für den firmeninternen Einsatz geeignet ist. Dazu werden in einem ersten Schritt Anforderungen der Process Owner und Systemadministrierenden gesammelt. Anschließend wird eine Testumgebung eingerichtet, um zu prüfen, inwieweit die Ansprüche des Unternehmens durch \glqq Icinga 2\grqq{} erfüllt werden können. Abschließend werden die Ergebnisse analysiert und eine Empfehlung ausgesprochen.

\subsection{Projektbegründung} 
\label{sec:Projektbegruendung}
Durch die zunehmende Digitalisierung aller Geschäftsprozesse sind Unternehmen äußerst abhängig von Computersystemen. Wird das Schutzziel der Verfügbarkeit nicht ausreichend gut verfolgt, kommt es zu Systemausfällen und der Geschäftsbetrieb ist nicht mehr möglich - mit unabsehbaren wirtschaftlichen Folgen. Um Ausfallsicherheit zu gewährleisten, ist ein umfangreiches und zuverlässiges Monitoring aller Ressourcen unerlässlich.

\subsection{Projektabgrenzung} 
\label{sec:Projektabgrenzung}
Als Monitoringsystem interagiert \glqq Icinga 2\grqq{} potenziell mit allen Geräten und Diensten im Netzwerk. Für dieses Projekt wird das Zusammenspiel auf virtuelle Maschinen mit ausgewählten Betriebssystemen und Diensten eingegrenzt.